\documentclass[11pt]{article}
\renewcommand{\baselinestretch}{1.05}
\usepackage{amsmath,amsthm,verbatim,amssymb,amsfonts,amscd, graphicx}
\usepackage{graphics}

\topmargin0.0cm
\headheight0.0cm
\headsep0.0cm
\oddsidemargin0.0cm
\textheight23.0cm
\textwidth16.5cm
\footskip1.0cm
\newcommand{\inchsign}{$^{\prime\prime}$}
\newcommand{\vep}[1]{\ensuremath{\varepsilon#1}}

 \begin{document}
 
\title{Problems Class \sc{iii}}
\author{Dr. James Mullaney}
\maketitle

\noindent
{\bf Equations and constants}

\noindent
The Friedmann Equation:
\begin{equation*}
    \left(\frac{\dot{a}}{a}\right)^2 = \frac{8\pi G}{3c^2}\varepsilon{}-\frac{\kappa c^2}{R_0^2}\frac{1}{a^2}
\end{equation*}

\noindent
The Fluid Equation:
\begin{equation*}
    \dot{\varepsilon{}}+3\frac{\dot{a}}{a}(\varepsilon{}+P) = 0
\end{equation*}

\noindent
Cosmological parameter values in The Benchmark Model:
\begin{equation*}
\Omega_{M,0} = 0.31,~\Omega_{D,0} = 0.69,~\Omega_{R,0} = 9\times10^{-5},~H_0 = 67.7~{\rm km~s^{-1}~Mpc^{-1}}
\end{equation*}

\noindent
Parsec in SI units: ${\rm 1~pc = 3.09\times10^{16}~m}$\\

\noindent
{\bf Prologue}\\
\noindent
The idea behind this problems class is to get you more familiar with working with the Friedmann Equation, and deriving its different (but equivalent) forms.\\

\noindent
{\bf Questions}

\begin{enumerate}
    \item Derive the expression for $\epsilon_{\rm m}$, the evolving matter energy density, in terms of scale factor, $a(t)$, the current critical energy density, $\epsilon_{\rm c,0}$, and the current matter parameter, $\Omega_{\rm m,0}$.
    \item Write the expressions for the evolving radiation energy and dark energy densities, $\epsilon_{\rm p}$ and $\epsilon_{\rm D}$, in the same respective terms.
    \item Derive the expression for $\frac{-\kappa c^2}{R_0^2}$ in terms of the Hubble constant, $H_0$, and the current energy parameter, $\Omega_0$.
    \item Putting the answers from all the above questions together, obtain the expression for $\dot{a}$ in terms of the Hubble constant, $H_0$, the various energy parameters, $\Omega_{\rm i, 0}$, and the scale factor, $a(t)$.
    \item And if we have time: by differentiating the answer you obtained in question 4 with respect to time, obtain the expresson for the current accelation of the Universe. Is the Universe currently accelerating, decelerating, or neither?    
\end{enumerate}
\end{document}